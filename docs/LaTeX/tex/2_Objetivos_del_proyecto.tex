\capitulo{2}{Objetivos del proyecto}

A continuación se detallan los objetivos que motivan la realización de este proyecto:

\section{Objetivos generales}\label{objetivos_generales}

\begin{itemize}
    \item Desarrollar un sistema \textit{software} que permita automatizar la evaluación de tareas de programación.
    \item Investigar la posibilidad de hacer uso de diferentes modelos de lenguaje a gran escala para la evaluación de estas tareas.
    \item Comprobar la posibilidad de delegar totalmente en estos modelos para la realización de estas tareas.
    \item Implementar operaciones e interacción con la aplicación:
    \begin{itemize}
        \item Permitir a los usuarios crear una cuenta.
        \item Permitir a los usuarios diseñar o definir los \textit{prompts} con los que interactuarán con los modelos.
        \item Permitir al usuario importar documentos en formato \textit{Markdown} con los contenidos de sus propias rúbricas.
        \item Permitir a los usuarios configurar diferentes parámetros del modelo que evaluará las tareas.
        \item Permitir a los usuarios ejecutar el modelo para que realice la evaluación.
        \item Permitir a los usuarios consultar el resultado de la ejecución de la evaluación.
        \item Permitir a los usuarios consultar datos sobre el modelo utilizado.
    \end{itemize}
\end{itemize}

\section{Objetivos técnicos}\label{objetivos_tecnicos}
\begin{itemize}
    \item Crear una aplicación web utilizando \textit{Python} y el \textit{framework} \textbf{Django}.
    \item Utilizar la arquitectura MVT (Model - View - Template) de Django.
    \item Utilizar una base de datos \textbf{PostgreSQL} para el almacenamiento de datos.
    \item Utilizar \textbf{Ollama} para la gestión de los modelos de lenguaje en entornos locales.
    \item Utilizar \textbf{Celery} y \textbf{Redis} para la gestión de tareas asíncronas.
    \item Utilizar \textbf{Docker} para la creación de contenedores y facilitar el despliegue del proyecto.
    \item Hacer uso de \textbf{Sphinx} para la generación de la documentación del proyecto.
    \item Gestionar el proyecto utilizando la herramienta de gestión \textbf{Zube} para la planificación y seguimiento.
    \item Realizar test con una cobertura que garanticen la calidad del producto final.
    \item Hacer uso del sistema de control de versiones \textbf{GIT} distribuido junto con la plataforma \textbf{Github}.
    \item Aplicar la metodología ágil \textbf{Scrum} para el desarrollo del proyecto.
    \item Utilizar herramientas de integración continua (CI/CD), como las proporcionadas por \textit{GitHub Actions}, para automatizar el despliegue del proyecto, la ejecución de pruebas, la medición de la cobertura de código con \textit{Coverage}, y el control de calidad integrando \textit{SonarCloud}.
\end{itemize}

\section{Objetivos personales}\label{objetivos_personasles}

El principal objetivo personal consiste en desarrollar una herramienta funcional, útil y escalable que no solo cumpla con los requisitos establecidos en la propuesta inicial, sino que también sea capaz de adaptarse a diferentes contextos educativos y evolucionar con facilidad. Se busca que el sistema tenga una aplicación práctica real, ofreciendo una solución eficiente y sostenible que pueda mantenerse y mejorarse en el tiempo.

Además, se plantean los siguientes objetivos específicos orientados al uso y aprendizaje de tecnologías clave durante el desarrollo del proyecto:

\begin{itemize}
    \item Utilizar el \textit{framework} \textbf{Django} para el desarrollo de aplicaciones web, configurándolo adecuadamente para trabajar con \textit{PostgreSQL} como sistema de bases de datos.
    \item Aprender a ejecutar modelos \textit{LLM} de forma local e integrarlos en una aplicación web.
    \item Investigar el comportamiento de los modelos con diferentes configuraciones para evaluar la viabilidad del proyecto.
    \item Conocer y utilizar \textbf{Docker} para la creación de contenedores, facilitando el despliegue del proyecto mediante \textbf{Docker Compose} como herramienta de orquestación.
    \item Profundizar en el uso de herramientas de control de versiones como \textbf{GIT} y plataformas como \textbf{GitHub}, empleando metodologías como \textit{GitFlow}.
    \item Aprender a utilizar herramientas de integración continua como \textbf{GitHub Actions}.
    \item Familiarizarse con herramientas de gestión de proyectos como \textbf{Zube}.
    \item Aprender a utilizar herramientas de análisis de calidad de código como \textbf{SonarCloud}, integrándolas en el ciclo de desarrollo.
    \item Aprender a generar documentación técnica automática mediante herramientas como \textbf{Sphinx}.
\end{itemize}
