\apendice{Plan de proyecto software}

\section{Introducción}

En este apartado se detalla la planificación del proyecto, una etapa clave que implica definir los objetivos y el alcance, identificar los requisitos y recursos necesarios, establecer un cronograma con hitos clave, asignar tareas al equipo y prever posibles riesgos. Además, se analiza el coste en términos de tiempo, recursos y, aunque se trate de un proyecto académico, el coste económico y sus posibles beneficios. Con los resultados obtenidos, se elabora la planificación temporal y se evalúa la viabilidad del proyecto.

\section{Planificación temporal}

En la gestión del proyecto se ha optado por seguir una metodología de desarrollo ágil \textit{Scrum} con algunas salvedades, al participar una persona con la supervisión del tutor y según la guía \textit{Scrum}: \textit{``Scrum emplea un enfoque iterativo e incremental para optimizar la previsibilidad y controlar el riesgo.
\textit{Scrum} involucra a grupos de personas que colectivamente tienen todas las habilidades y experiencia
para hacer el trabajo y compartir o adquirir tales habilidades según sea necesario.''}~\cite{scrumguides:guia}

Para el desarrollo incremental del proyecto se han planificado una serie de \textit{sprints} de dos semanas de duración, planificando al inicio de cada uno de ellos las tareas a desarrollar y revisando estas tareas a la finalización del mismo.

Para la gestión de los \textit{sprints}, planificación y asignación de tareas se usó la herramienta \textit{Zube}~\cite{web:zube}.

A continuación se muestran los \textit{sprints} llevados a cabo durante las distintas etapas de este proyecto.

\subsection{\emph{Sprint} 0 (27/02/2025 hasta 6/03/2025)}
Se define un \textit{sprint} inicial de una semana de duración en el que se realiza una primera aproximación a la propuesta de proyecto. En este \textit{sprint} se realizan las siguientes tareas:

\begin{itemize}
    \item Entrevista con tutor.
    \item Preparación de los entornos \textit{Github, Zube}.
    \item Investigación sobre herramientas.
    \item Investigación sobre \textit{LLMs}.
    \item Inicio de la documentación.
    \item Configuración del entorno de desarrollo.
    \item Definición de objetivos generales.
    \item Diseño de caso de uso genérico.
\end{itemize}

A continuación se muestra el gráfico \textit{burndown} del \textit{sprint}.
\imagen{sprint0}{Gráfico \textit{burndown} del \textit{sprint} 0}

\subsection{\emph{Sprint} 1 (7/03/2025 hasta 20/03/2025)}
Segundo \textit{sprint} en el que se realiza una planificación más específica de las tareas:
\begin{itemize}
    \item Implementación de \textit{Login}.
    \item Creación de un \textit{mockup} del esqueleto de la página \textit{Home}.
    \item Realización de pruebas con distintos \textit{LLMs}.
    \item Revisión y actualización del caso de uso general.
    \item Diseño de diagramas ER.
    \item Personalización del modelo de usuario.
    \item Implementación del registro de usuarios.
\end{itemize}


Gráfico \textit{burndown} del \textit{sprint}.
\imagen{sprint1}{Gráfico \textit{burndown} del \textit{sprint} 1}

\subsection{\emph{Sprint} 2 (20/03/2025 hasta 3/04/2025)}
A continuación se muestran las tareas planificadas:
\begin{itemize}
    \item Implementación de \textit{Home page}.
    \item Implementación inicial de la sección rúbricas.
    \item Implementación inicial de la sección \textit{prompts}.
    \item Integración de \textit{SonarQube cloud}.
    \item Revisión de \textit{Login} y Registro de usuarios.
\end{itemize}

Gráfico \textit{burndown} del \textit{sprint}.
\imagen{sprint2}{Gráfico \textit{burndown} del \textit{sprint} 2}

\subsection{\emph{Sprint} 3 (03/04/2025 hasta 16/04/2025)}
A continuación se muestran las tareas planificadas:
\begin{itemize}
    \item Implementación de modelo Correcciones.
    \item Implementación de formulario para el modelo.
    \item Implementación inicial de la carga de ficheros.
    \item Implementación de \textit{test}.
    \item Integración de las vistas.
    \item Revisión de Rúbricas y \textit{Prompts}.
\end{itemize}

Gráfico \textit{burndown} del \textit{sprint}.
\imagen{sprint3}{Gráfico \textit{burndown} del \textit{sprint} 3}

\subsection{\emph{Sprint} 4 (16/04/2025 hasta 30/04/2025)}
Tareas planificadas:
\begin{itemize}
    \item Investigación sobre integración de modelo en la aplicación.
    \item Integración de modelo en la aplicación.
    \item Investigación sobre implementación de respuesta asíncrona.
    \item Implementación de respuesta asíncrona.
\end{itemize}

Gráfico \textit{burndown} del \textit{sprint}.
\imagen{sprint4}{Gráfico \textit{burndown} del \textit{sprint} 4}

\subsection{\emph{Sprint} 5 (1/05/2025 hasta 15/05/2025)}
Tareas planificadas para este \textit{sprint}:
\begin{itemize}
    \item Comprobación y generación de documentación.
    \item Lanzar \textit{Release}.
    \item Modificación de orden del menú.
    \item Parametrización del modelo.
    \item Completar sección \textit{LLM}.
\end{itemize}

Gráfico \textit{burndown} del \textit{sprint}.
\imagen{sprint5}{Gráfico \textit{burndown} del \textit{sprint} 5}

\subsection{\emph{Sprint} 6 (16/05/2025 hasta 29/05/2025)}
A continuación se muestran las tareas planificadas:
\begin{itemize}
    \item Internacionalización de la aplicación.
    \item Añadir cabecera con campo de búsqueda en las tablas.
    \item Revisar documentación generada con \textit{Sphinx}.
    \item Avanzar documentación de la memoria.
\end{itemize}

Gráfico \textit{burndown} del \textit{sprint}.
\imagen{sprint6}{Gráfico \textit{burndown} del \textit{sprint} 6}

\subsection{\emph{Sprint} 7 (30/05/2025 hasta 05/06/2025)}
Tareas planificadas:
\begin{itemize}
    \item Revisión de proyecto.
    \item Avance de anexos.
    \item Avanzar memoria.
\end{itemize}

Gráfico \textit{burndown} del \textit{sprint}.
\imagen{sprint7}{Gráfico \textit{burndown} del \textit{sprint} 7}

\subsection{\emph{Sprint} 8 (5/06/2025 hasta 11/06/2025)}
Tareas planificadas:
\begin{itemize}
    \item Preparar presentación.
    \item Actualizar proyecto.
    \item Grabar vídeos.
    \item Revisar documentación.
\end{itemize}

Gráfico \textit{burndown} del \textit{sprint}.
\imagen{sprint8}{Gráfico \textit{burndown} del \textit{sprint} 8}

\section{Estudio de viabilidad}

\subsection{Viabilidad económica}

En este apartado se realiza una estimación de los costes del proyecto, lo más aproximada posible a un entorno empresarial, y se analizan los posibles beneficios en caso de que el proyecto se comercialice.

Los costes principales son de personal, junto con \textit{hardware}, \textit{software} y, si aplica, infraestructura.

\subsubsection{Estudio de costes}

El proyecto ha tenido una duración de cuatro meses y se ha desarrollado por una persona, incluye formación, 
implementación y documentación. 

Teniendo en cuenta el salario medio de un programador junior en España~\cite{web:talen} y 
considerando ausencia de deducciones tributarias.

\tablaSmallSinColores{Costes de personal}{l|r|r}{costes_personal}{
\textbf{Concepto} & \textbf{Coste anual} & \textbf{Prorrateo (4 meses)}\\
}{
\textit{Salario bruto} & 21.000,00€ & 7.000,00€\\
\textit{Retención IRPF (17 \%)} & 3.570,00€ & 1.190,00€\\
\textit{Seguridad Social (7 \%)} & 1.220,10€ & 406,70€\\
\textit{Salario neto} & 16.209,90€ & 5.403,30€\\ 
}

A continuación, se detallan los costes asociados al \textit{hardware} y \textit{software} utilizados en el proyecto. Para su desarrollo, 
se empleó un \textit{MacBook Pro}, cuyo sistema operativo (macOS) cuenta con una licencia incluida, y como entorno de desarrollo integrado (\textit{IDE}) 
se utilizó \textit{Visual Studio Code}, el cual es gratuito.

Dado que el proyecto tiene una duración de 4 meses, se ha aplicado un criterio de amortización proporcional del \textit{hardware}. En concreto, se ha considerado una vida útil del equipo de 3 años, imputando únicamente el coste correspondiente a los 4 meses de uso.

\tablaSmallSinColores{Costes de \textit{hardware} y \textit{software}}{l|r|r}{costes_hw_sw}{
\textbf{Concepto} & \textbf{Coste total} & \textbf{Coste amortizado} \\
}{
\textit{Ordenador personal} & 1.500,00 € & 166,67 € \\
\textit{macOS } & 0,00 € & 0,00 € \\
\textit{Visual Studio Code } & 0,00 € & 0,00 € \\
}
También se consideran otros costes, como el consumo eléctrico y otros gastos operativos derivados del desarrollo y ejecución del proyecto, 
tales como servicios de conectividad y material de oficina, entre otros.

\tablaSmallSinColores{Costes varios}{l|r|r}{costes_varios}{
\textbf{Concepto} & \textbf{Coste mensual} & \textbf{Prorrateo (4 meses)}\\
}{
\textit{Consumo eléctrico} & 100,00 € & 400,00 € \\
\textit{Internet} & 30,00 € & 120,00 €\\
\textit{Material de oficina} & 30,00 € & 30,00€ \\
}

Los costes estimados totales son:

\tablaSmallSinColores{Costes totales}{l|r}{costes_totales}{
\textbf{Concepto} & \textbf{Coste} \\
}{
\textit{Costes de personal} & 7.000,00 € \\
\textit{Costes \textit{hardware} y \textit{software}} & 166,67 € \\
\textit{Costes varios} & 550,00 € \\
\hline
\textit{\textbf{TOTAL}} & 7.716,67 € \\
}

\subsection{Análisis de beneficios}

Este proyecto no se plantea inicialmente como un mecanismo para generar beneficios económicos, sino como una herramienta gratuita destinada 
a facilitar la corrección y evaluación automática de tareas, especialmente en el ámbito de la programación.

No obstante, en caso de considerar su monetización, existen diversas vías viables. 
Una opción sería incluir publicidad contextual no intrusiva en la interfaz web, lo cual permitiría generar 
ingresos suficientes para cubrir los costes operativos del proyecto. Otra posibilidad consiste en ofrecer 
servicios personalizados bajo demanda, como adaptaciones específicas del sistema para centros educativos o 
empresas de formación técnica. Asimismo, se podría licenciar el uso de la plataforma a instituciones interesadas 
en integrar esta herramienta en sus propios entornos virtuales de aprendizaje, ya sea mediante \textit{APIs} o despliegues locales.

A continuación se presenta una estimación aproximada de cuánto podría cobrarse en cada caso:
\begin{enumerate}
    \item \textbf{Publicidad contextual no intrusiva}: la integración de anuncios discretos en la interfaz web podría generar ingresos complementarios. Estos ingresos pueden variar ampliamente según el tráfico.
    
    \item \textbf{Servicios personalizados bajo demanda}: la adaptación del sistema a necesidades específicas de centros educativos o entidades formativas podría suponer ingresos entre \textbf{500 y 2.000€ por proyecto}. Con apenas \textbf{2 o 3 contrataciones anuales}, se cubriría una parte significativa de los costes de desarrollo y soporte técnico.
    
    \item \textbf{Licencias institucionales}: ofrecer el sistema como un servicio licenciado a instituciones interesadas (con acceso vía \textit{API} o instalación local) permitiría ingresos más estables. Una tarifa anual por uso institucional podría oscilar entre \textbf{1.000 y 3.000€}, dependiendo del grado de personalización y soporte requerido.
\end{enumerate}

\subsection{Viabilidad legal}

\subsubsection{Licencias de \textit{software}}

En este apartado se detallan las licencias de las herramientas y bibliotecas de software utilizadas durante el desarrollo del proyecto. Asimismo, se asignará una licencia al propio proyecto.

En la tabla siguiente se enumeran las herramientas y bibliotecas de software empleadas, junto con sus respectivas licencias. Solo se incluyen aquellas que forman parte del entorno de ejecución del sistema o que desempeñan un papel técnico relevante.

\tablaSmallSinColores{Licencias de terceros}{l|r}{licencias_terceros}{
\textbf{Herramienta} & \textbf{Licencia} \\
}{
\textit{Django} & BSD \\
\textit{djangorestframework} & BSD \\
\textit{python-decouple} & MIT \\
\textit{psycopg2-binary} & BSD-style \\
\textit{sqlparse} & BSD \\
\textit{langchain} & MIT \\
\textit{langchain-community} & MIT \\
\textit{langchain-ollama} & MIT \\
\textit{ollama} & MIT \\
\textit{celery} & BSD \\
\textit{redis} & BSD (v5.2.1) \\
\textit{sphinx} & BSD \\
\textit{sphinx-book-theme} & MIT \\
\textit{Tailwind CSS} & MIT \\
\textit{Flowbite} & MIT \\
}

Todas las licencias listadas son de tipo permisivo (BSD o MIT)~\cite{wiki:licenciaBSD,wiki:licenciaMIT}, lo que garantiza su compatibilidad con la licencia \textbf{MIT} seleccionada para este proyecto.

Para la licencia de este proyecto se ha optado por la \textit{MIT License}, una licencia de \textit{software} libre y permisiva que permite a los usuarios utilizar, copiar, modificar y distribuir el código de manera sencilla y flexible. Esta elección facilita la colaboración y la adopción del \textit{software} en diferentes entornos educativos y profesionales.

Además, la \textit{MIT License} es compatible con todas las licencias de las herramientas y bibliotecas empleadas en el desarrollo del proyecto, lo que garantiza la coherencia y legalidad en el uso conjunto de estas tecnologías.
