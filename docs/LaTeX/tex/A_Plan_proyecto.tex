\apendice{Plan de Proyecto Software}

\section{Introducción}

\section{Planificación temporal}
En la gestión del proyecto se ha optado por seguir una metodología de desarrollo ágil \textit{Scrum} con algunas salvedades, al participar una persona con la supervisión del tutor y según la guía Scrum: \textit{"Scrum emplea un enfoque iterativo e incremental para optimizar la previsibilidad y controlar el riesgo.
Scrum involucra a grupos de personas que colectivamente tienen todas las habilidades y experiencia
para hacer el trabajo y compartir o adquirir tales habilidades según sea necesario."}~\cite{scrumguides:guia}

Para el desarrollo incremental del proyecto se han planificado una serie de sprints de dos semanas de duración, planificando al inicio de cada uno de ellos las tareas a desarrollar y revisando estas tareas a la finalización del mismo.

Para la gestión de los sprints, planificación y asignación de tareas se usó la herramienta \textit{Zube}

A continuación se muestran los \textit{sprints} llevados a cabo durante las distintas etapas de este proyecto.

\subsection{\emph{Sprint} 0 (27/02/2025 hasta 6/03/2025)}
Se define un sprint inicial de una semana de duración en el que se realiza una primera aproximación a la propuesta de proyecto. En este sprint se realizan las siguientes tareas:

\begin{itemize}
    \item Entrevista con tutor.
    \item Preparación de los entornos Github, Zube.
    \item Investigación sobre herramientas.
    \item Investigación sobre LLMs.
    \item Inicio de la documentación.
    \item Configuración del entorno de desarrollo.
    \item Definición de objetivos generales.
    \item Diseño de caso de uso genérico.
\end{itemize}

\subsection{\emph{Sprint} 1 (7/03/2025 hasta 20/03/2025)}
Segundo sprint en el que se realiza una planificación más específica de las tareas:
\begin{itemize}
    \item Implementación de Login.
    \item Creación de un mockup del esqueleto de la página Home.
    \item Realización de pruebas con distintos LLMs.
    \item Revisión y actualización del caso de uso general.
    \item Diseño de diagramas ER.
    \item Personalización del modelo de usuario.
    \item Implementación del registro de usuarios.
\end{itemize}

\subsection{\emph{Sprint} 2 (20/03/2025 hasta 3/04/2025)}
A continuación se muestran las tareas planificadas:
\begin{itemize}
    \item Implementación de Home page.
    \item Implementación inicial de la sección rúbricas.
    \item Implementación inicial de la sección prompts.
    \item Integración de SonarQube cloud.
    \item Revisión de Login y Registro de usuarios.
\end{itemize}

\subsection{\emph{Sprint} 3 (03/04/2025 hasta 16/04/2025)}
A continuación se muestran las tareas planificadas:
\begin{itemize}
    \item Implementación de modelo Correcciones.
    \item Implementación de formulario para el modelo.
    \item Implementación inicial de la carga de ficheros.
    \item Implementación de test.
    \item Integración de las vistas.
    \item Revisión de Rúbricas y Prompts.
\end{itemize}


\section{Estudio de viabilidad}

\subsection{Viabilidad económica}

\subsection{Viabilidad legal}


