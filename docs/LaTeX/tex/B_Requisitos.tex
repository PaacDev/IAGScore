\apendice{Especificación de Requisitos}

\section{Introducción}

En este anexo se describen los servicios que debe ofrecer la aplicación, para la configuración de correcciones y la evaluación de tareas, principalmente de programación. También se detallan las restricciones asociadas a su funcionamiento. La función principal de esta especificación de requisitos es servir como medio de comunicación claro y preciso entre clientes, usuarios y desarrolladores.

Se incluirán tanto los requisitos funcionales como los no funcionales del sistema.

\section{Objetivos generales}

Este trabajo se ha realizado persiguiendo el cumplimiento de los siguientes \textbf{objetivos:}

\begin{itemize}
    \item Desarrollar un sistema \textit{software} que permita a los usuarios la corrección de ejercicios de programación.
    \item El usuario debe estar \textit{registrado} para acceder al sistema iniciando sesión.
    \item Los usuarios \textit{importarán} sus propias rúbricas en formato \textit{Markdown}.
    \item Las rúbricas importadas en \textit{Markdown} serán visibles en formato \textit{HTML}.
    \item Los usuarios crearán sus propios \textit{prompts}.
    \item La evaluación de los ejercicios se realizará mediante modelos de lenguaje \textit{LLM}.
    \item Los usuarios podrán configurar aspectos determinados del \textit{LLM} usado.
\end{itemize}

\section{Catálogo de requisitos}
Requisitos funcionales y no funcionales para cumplir con los objetivos generales.

\subsection{Requisitos funcionales}
\begin{itemize}
    \item \textbf{RF-1 - Registro de usuarios.} Los usuarios podrán registrarse mediante nombre, correo electrónico y contraseña.
    \item \textbf{RF-2 - \textit{Login} de usuarios.} Los usuarios deberán iniciar sesión para usar la aplicación.
    \item \textbf{RF-3 Crear \textit{prompts}.} Los usuarios crearán sus propios \textit{prompts}.
    \begin{itemize}
        \item \textbf{RF-3.1 - Listar \textit{prompts}.} Los usuarios podrán consultar sus \textit{prompts} creados.
        \item \textbf{RF-3.2 - Mostrar \textit{prompt}.} Los usuarios podrán visualizar un \textit{prompt} concreto.
        \item \textbf{RF-3.3 - Eliminar \textit{prompt}.} Los usuarios podrán eliminar un \textit{prompt}.
    \end{itemize}
    \item \textbf{RF-4 - Importar rúbricas.} Los usuarios podrán importar sus propias rúbricas en formato \textit{Markdown}.
    \begin{itemize}
        \item \textbf{RF-4.1 - Listar rúbricas.} Los usuarios podrán consultar sus rúbricas importadas.
        \item \textbf{RF-4.2 - Mostrar rúbrica.} Los usuarios podrán visualizar los detalles de una rúbrica concreta en formato \textit{HTML}.
        \item \textbf{RF-4.3 - Eliminar rúbrica.} Los usuarios podrán eliminar una rúbrica importada.
    \end{itemize}
    \item \textbf{RF-5 Configurar correcciones.} Los usuarios configurarán las correcciones.
    \begin{itemize}
        \item \textbf{RF-5.1 - Listar correcciones.} Los usuarios podrán consultar sus correcciones configuradas.
        \item \textbf{RF-5.2 - Mostrar configuración.} Los usuarios podrán visualizar los detalles de configuración de una corrección.
        \item \textbf{RF-5.3 - Eliminar corrección.} Los usuarios podrán eliminar una corrección.
        \item \textbf{RF-5.4 - Ejecutar corrección.} Los usuarios podrán ejecutar una corrección.
        \item \textbf{RF-5.5 - Descargar corrección.} Los usuarios podrán descargar el resultado de la corrección ejecutada.
    \end{itemize}

\end{itemize}

\section{Especificación de requisitos}

A continuación se muestran los casos de uso contemplados para el desarrollo del proyecto 
\newpage
\subsection{Casos de uso}

\imagen{caso_de_uso_general}{Diagrama de casos de uso}

% Caso de Uso 1 -> Registrar usuario.
\begin{table}[p]
	\centering
	\begin{tabularx}{\linewidth}{ p{0.21\columnwidth} p{0.71\columnwidth} }
		\toprule
		\textbf{CU-1}    & \textbf{Registro de usuario}\\
		\toprule
		\textbf{Versión}              & 1.0    \\
		\textbf{Autor}                & Pedro Antonio Abellaneda Canales \\
		\textbf{Requisitos asociados} & RF-1 \\
		\textbf{Descripción}          & Permite registrase a un usuario \\
		\textbf{Precondición}         & El usuario no puede estar \textit{logeado} \\
		\textbf{Acciones}             &
		\begin{enumerate}
			\def\labelenumi{\arabic{enumi}.}
			\tightlist
			\item El usuario entra en la aplicación y accede a la página de registro
			\item El usuario introduce su email y una contraseña
            \item El usuario confirma el registro con el botón registrar
		\end{enumerate}\\
		\textbf{Postcondición}        & \begin{itemize}
                                        \tightlist
		                                \item La aplicación redirige a la página de \textit{login}.
                                        \item Mensaje: Usuario creado correctamente.
		                              \end{itemize}  \\ 
                                      
		\textbf{Excepciones}          & \begin{itemize}
                                        \tightlist
		                                \item Mensaje: Ya existe Usuario con este Email.
		                              \end{itemize}  \\ 
		\textbf{Importancia}          & Alta \\
		\bottomrule
	\end{tabularx}
	\caption{CU-1 Registrar usuario.}
\end{table}


% Caso de Uso 2 -> Login de usuario.
\begin{table}[p]
	\centering
	\begin{tabularx}{\linewidth}{ p{0.21\columnwidth} p{0.71\columnwidth} }
		\toprule
		\textbf{CU-2}    & \textbf{\textit{Login} de usuario} \\
		\midrule
		\textbf{Versión}              & 1.0    \\
		\textbf{Autor}                & Pedro Antonio Abellaneda Canales \\
		\textbf{Requisitos asociados} & RF-2 \\
		\textbf{Descripción}          & Permite \textit{logearse} a un usuario \\
		\textbf{Precondición}         & El usuario debe estar registrado \\
		\textbf{Acciones}             &
		\begin{enumerate}
			\def\labelenumi{\arabic{enumi}.}
			\tightlist
			\item El usuario entra en la aplicación y accede a la página de \textit{login}.
			\item El usuario introduce su \textit{email} y una contraseña.
			\item El usuario confirma pulsando el botón entrar.
		\end{enumerate} \\
		\textbf{Postcondición}        & \begin{itemize}
                                        \tightlist
		                                \item La aplicación redirige a la página \textit{home}.
		                              \end{itemize}  \\
        
		\textbf{Excepciones}          & \begin{itemize}
                                        \tightlist
		                                \item Mensaje: Usuario o contraseña incorrectos
		                              \end{itemize}  \\
		\textbf{Importancia}          & Alta \\
		\bottomrule
	\end{tabularx}
	\caption{CU-2 Login de usuario.}
	\label{tab:CU-2}
\end{table}

% Caso de Uso 3 -> Crear Prompts
\begin{table}[p]
	\centering
	\begin{tabularx}{\linewidth}{ p{0.21\columnwidth} p{0.71\columnwidth} }
		\toprule
		\textbf{CU-3}    & \textbf{Crear \textit{Prompts}} \\
		\midrule
		\textbf{Versión}              & 1.0    \\
		\textbf{Autor}                & Pedro Antonio Abellaneda Canales \\
		\textbf{Requisitos asociados} & RF-3 \\
		\textbf{Descripción}          & Permite crear un \textit{prompt} a un usuario \\
		\textbf{Precondición}         & El usuario debe estar \textit{logeado} \\
		\textbf{Acciones}             &
		\begin{enumerate}
			\def\labelenumi{\arabic{enumi}.}
			\tightlist
			\item El usuario accede a la sección de \textit{prompts} y pulsa botón nuevo \textit{prompt}.
            \item El usuario introduce un nombre.
			\item El usuario introduce el texto del \textit{prompt}.
            \item El usuario pulsa botón importar.
		\end{enumerate} \\
		\textbf{Postcondición}        
                                    & 
                                    \begin{itemize} 
                                        \tightlist
                                        \item La aplicación redirecciona a la página de \textit{prompts} 
                                        \item  Mensaje: \textit{Prompt} creado correctamente 
                                    \end{itemize}  
                                    \\ 
		\textbf{Excepciones}          
                                        & 
                                        \begin{itemize}
                                        \tightlist
                                            \item Mensaje: Error al guardar el prompt: Prompt ya existente
                                        \end{itemize}  
                                        \\ 
		\textbf{Importancia}          & Alta \\
		\bottomrule
	\end{tabularx}
	\caption{CU-3 Crear Prompts.}
	\label{tab:CU-3}
\end{table}

% Caso de Uso 4 -> Listar Prompts
\begin{table}[p]
	\centering
	\begin{tabularx}{\linewidth}{ p{0.21\columnwidth} p{0.71\columnwidth} }
		\toprule
		\textbf{CU-4}    & \textbf{Listar \textit{Prompts}} \\
		\midrule
		\textbf{Versión}              & 1.0    \\
		\textbf{Autor}                & Pedro Antonio Abellaneda Canales \\
		\textbf{Requisitos asociados} & RF-3.1 \\
		\textbf{Descripción}          & Permite listar los \textit{prompts} de un usuario \\
		\textbf{Precondición}         
                                    & 
                                    \begin{itemize}
                                        \tightlist
                                        \item El usuario debe estar logeado.
                                        \item  El usuario debe haber creado algún \textit{prompt}.
                                    \end{itemize}  
                                    \\
		\textbf{Acciones}             &
		\begin{enumerate}
			\def\labelenumi{\arabic{enumi}.}
			\tightlist
			\item El usuario accede a la sección de prompts.
		\end{enumerate} \\
		\textbf{Postcondición}        
                                & 
                                \begin{itemize} 
                                    \tightlist
                                    \item Se muestra la tabla con nombre y fecha de creación y acciones disponibles de cada una de los prompts creados.
                                \end{itemize}  
                                \\ 
		\textbf{Excepciones}          
        & 
        \\ 
		\textbf{Importancia}          & Alta \\
		\bottomrule
	\end{tabularx}
	\caption{CU-4 Listar Prompts.}
	\label{tab:CU-4}
\end{table}

% Caso de Uso 5 -> Mostrar Prompt
\begin{table}[p]
	\centering
	\begin{tabularx}{\linewidth}{ p{0.21\columnwidth} p{0.71\columnwidth} }
		\toprule
		\textbf{CU-5}    & \textbf{Mostrar \textit{Prompt}} \\
		\midrule
		\textbf{Versión}              & 1.0    \\
		\textbf{Autor}                & Pedro Antonio Abellaneda Canales \\
		\textbf{Requisitos asociados} & RF-3.2 \\
		\textbf{Descripción}          & Permite mostrar un \textit{prompt} concreto \\
		\textbf{Precondición}         
                                    & 
                                    \begin{itemize}
                                        \tightlist
                                        \item El usuario debe estar logeado.
                                        \item  El usuario debe haber creado el \textit{prompt}.
                                    \end{itemize}  
                                    \\
		\textbf{Acciones}             &
		\begin{enumerate}
			\def\labelenumi{\arabic{enumi}.}
			\tightlist
			\item El usuario accede a la sección de \textit{prompts}.
            \item El usuario busca en la tabla el \textit{prompt} que desee.
			\item El usuario pulsa el botón Mostrar del \textit{prompt} que desea visualizar.
		\end{enumerate} \\
		\textbf{Postcondición}        
                                & 
                                \begin{itemize} 
                                    \tightlist
                                    \item Se muestra el texto del \textit{prompt}.
                                    \item Se muestran botones para volver o eliminar.
                                \end{itemize}  
                                \\ 
		\textbf{Excepciones}          
        & 
                                        \begin{itemize}
                                        \tightlist
                                            \item Error 404 – Prompt no encontrado.
                                        \end{itemize}  
                                        \\ 
		\textbf{Importancia}          & Alta \\
		\bottomrule
	\end{tabularx}
	\caption{CU-5 Mostrar \textit{Prompts}.}
	\label{tab:CU-5}
\end{table}


% Caso de Uso 6 -> Eliminar Prompt
\begin{table}[p]
	\centering
	\begin{tabularx}{\linewidth}{ p{0.21\columnwidth} p{0.71\columnwidth} }
		\toprule
		\textbf{CU-6}    & \textbf{Eliminar \textit{Prompt}} \\
		\midrule
		\textbf{Versión}              & 1.0    \\
		\textbf{Autor}                & Pedro Antonio Abellaneda Canales \\
		\textbf{Requisitos asociados} & RF-3.3 \\
		\textbf{Descripción}          & Permite eliminar un \textit{prompt} concreto \\
		\textbf{Precondición}         
                                    & 
                                    \begin{itemize}
                                        \tightlist
                                        \item El usuario debe estar autenticado.
                                        \item  El usuario debe haber creado el \textit{prompt}.
                                    \end{itemize}  
                                    \\
		\textbf{Acciones}             &
                                    \begin{enumerate}
                                        \tightlist
                                        \item Opción 1 - Sin visualizar contenido:
                            		\begin{enumerate}
                            			\def\labelenumi{\arabic{enumi}.}
                            			\tightlist
                            			\item El usuario accede a la sección de \textit{prompts}.
                                        \item El usuario busca en la tabla el \textit{prompt} que desee.
                            			\item El usuario pulsa el botón eliminar.
                                        \item El usuario confirma la opción eliminar.
                            		\end{enumerate}
                                        \item Opción 2 - Visualizando contenido:
                            		\begin{enumerate}
                            			\def\labelenumi{\arabic{enumi}.}
                            			\tightlist
                            			\item El usuario accede a la sección de \textit{prompts}.
                                        \item El usuario busca en la tabla el \textit{prompt} que desee.
                            			\item El usuario pulsa el botón mostrar.
                                        \item El usuario visualiza el contenido del \textit{prompt}.
                                        \item El usuario pulsa el botón eliminar.
                                        \item El usuario confirma que desea eliminar.
                                    \end{enumerate}
                            		\end{enumerate} \\
		\textbf{Postcondición}        
                                & 
                                \begin{itemize} 
                                    \tightlist
                                    \item El \textit{prompt} es eliminado de la base de datos.
                                    \item La aplicación redirige a la vista de \textit{prompts}.
                                    \item Mensaje: Prompt eliminado correctamente.
                                \end{itemize}  
                                \\ 
		\textbf{Excepciones}          
                                    & 
                                        \begin{itemize}
                                        \tightlist
                                            \item Error 404 - Prompt no encontrado
                                        \end{itemize}  
                                        \\ 
		\textbf{Importancia}          & Alta \\
		\bottomrule
	\end{tabularx}
	\caption{CU-6 Eliminar \textit{Prompts}.}
	\label{tab:CU-6}
\end{table}


% Caso de Uso 7 -> Importar rúbricas
\begin{table}[p]
	\centering
	\begin{tabularx}{\linewidth}{ p{0.21\columnwidth} p{0.71\columnwidth} }
		\toprule
		\textbf{CU-7}    & \textbf{Importar rúbrica} \\
		\midrule
		\textbf{Versión}              & 1.0    \\
		\textbf{Autor}                & Pedro Antonio Abellaneda Canales \\
		\textbf{Requisitos asociados} & RF-4 \\
		\textbf{Descripción}          & Permite importar una rúbrica a un usuario \\
		\textbf{Precondición}         & El usuario debe estar autenticado \\
		\textbf{Acciones}             &
		\begin{enumerate}
			\def\labelenumi{\arabic{enumi}.}
			\tightlist
			\item El usuario accede a la sección de rúbricas y pulsa botón nueva rúbrica.
			\item El usuario introduce un nombre para la rúbrica.
			\item El usuario importa una rúbrica en formato Markdown.
		\end{enumerate} \\
		\textbf{Postcondición}      
                                & 
                                \begin{itemize} 
                                    \tightlist
                                    \item La aplicación redirige a la página de vista de rúbricas importadas.
                                    \item Mensaje: Rúbrica importada correctamente.
                                \end{itemize}  
                                \\   
		\textbf{Excepciones}         
                                        & 
                                        \begin{itemize}
                                        \tightlist
                                            \item Mensaje: Error al guardar la rúbrica: Rúbrica ya existente.
                                            \item Mensaje: El archivo debe estar codificado en UTF-8.
                                            \item Mensaje: El archivo subido no es un archivo Markdown.
                                        \end{itemize}  
                                        \\                       
		\textbf{Importancia}          & Alta \\
		\bottomrule
	\end{tabularx}
	\caption{CU-7 Importar rúbrica.}
	\label{tab:CU-7}
\end{table}

% Caso de Uso 8 -> Listar rúbricas
\begin{table}[p]
	\centering
	\begin{tabularx}{\linewidth}{ p{0.21\columnwidth} p{0.71\columnwidth} }
		\toprule
		\textbf{CU-8}    & \textbf{Listar rúbricas} \\
		\midrule
		\textbf{Versión}              & 1.0    \\
		\textbf{Autor}                & Pedro Antonio Abellaneda Canales \\
		\textbf{Requisitos asociados} & RF-4.1 \\
		\textbf{Descripción}          & Permite ver las rúbricas del usuario \\
		\textbf{Precondición}          & \begin{itemize}
                                        \tightlist
		                                  \item El usuario debe estar autenticado.
		                                  \item El usuario debe haber importado una rúbrica.
		                                 \end{itemize} \\
		\textbf{Acciones}             &
		\begin{enumerate}
			\def\labelenumi{\arabic{enumi}.}
			\tightlist
			\item El usuario accede a la sección de rúbricas
		\end{enumerate} \\
		\textbf{Postcondición}        & \begin{itemize}
                                        \tightlist
		                                  \item Se muestra la tabla con nombre, fecha de creación y acciones para cada una de las rúbricas importadas.		           
		                                 \end{itemize} \\ 
		\textbf{Excepciones}          & \\ 
		\textbf{Importancia}          & Alta \\
		\bottomrule
	\end{tabularx}
	\caption{CU-8 Listar rúbricas.}
	\label{tab:CU-8}
\end{table}

% Caso de Uso 9 -> Mostrar rúbrica
\begin{table}[p]
	\centering
	\begin{tabularx}{\linewidth}{ p{0.21\columnwidth} p{0.71\columnwidth} }
		\toprule
		\textbf{CU-9}    & \textbf{Mostrar rúbrica} \\
		\midrule
		\textbf{Versión}              & 1.0    \\
		\textbf{Autor}                & Pedro Antonio Abellaneda Canales \\
		\textbf{Requisitos asociados} & RF-4.2 \\
		\textbf{Descripción}          & Permite mostrar una rúbrica a un usuario \\
		\textbf{Precondición}         & \begin{itemize}
                                        \tightlist
		                                  \item El usuario debe estar autenticado.
		                                  \item El usuario debe haber importado una rúbrica.
		                                 \end{itemize} \\
		\textbf{Acciones}             &
		\begin{enumerate}
			\def\labelenumi{\arabic{enumi}.}
			\tightlist
			\item El usuario accede a la sección de rúbricas.
            \item El usuario busca en la tabla la rúbrica que desee.
			\item El usuario pulsa el botón Mostrar de la rúbrica que desee.
		\end{enumerate} \\
		\textbf{Postcondición}        & \begin{itemize}
                                        \tightlist
		                                  \item La aplicación redirige a la página de vista de una rúbrica concreta. 
                                          \item se muestra el contenido de la rúbrica en formato \textit{HTML}.
		                                 \end{itemize} \\
		\textbf{Excepciones}          & \begin{itemize}
                                        \tightlist
		                                  \item Error 404 - Rúbrica no encontrada.
		                                 \end{itemize} \\
		\textbf{Importancia}          & Alta \\
		\bottomrule
	\end{tabularx}
	\caption{CU-9 Mostrar rúbrica.}
	\label{tab:CU-9}
\end{table}

% Caso de Uso 10 -> Eliminar rúbrica
\begin{table}[p]
	\centering
	\begin{tabularx}{\linewidth}{ p{0.21\columnwidth} p{0.71\columnwidth} }
		\toprule
		\textbf{CU-10}    & \textbf{Eliminar rúbrica} \\
		\midrule
		\textbf{Versión}              & 1.0    \\
		\textbf{Autor}                & Pedro Antonio Abellaneda Canales \\
		\textbf{Requisitos asociados} & RF-4.3 \\
		\textbf{Descripción}          & Permite eliminar una rúbrica a un usuario \\
		\textbf{Precondición}         & \begin{itemize}
                                        \tightlist
		                                  \item El usuario debe estar autenticado.
		                                  \item El usuario debe haber importado una rúbrica.
		                                 \end{itemize} \\
		\textbf{Acciones}             &
                                    \begin{enumerate}
                                        \tightlist
                                        \item Opción 1 - Sin visualizar contenido:
                            		\begin{enumerate}
                            			\def\labelenumi{\arabic{enumi}.}
                            			\tightlist
                            			\item El usuario accede a la sección de rúbricas.
                                        \item El usuario busca en la tabla la rúbrica que desee.
                            			\item El usuario pulsa el botón eliminar.
                                        \item El usuario confirma la opción eliminar.
                            		\end{enumerate}
                                        \item Opción 2 - Visualizando contenido:
                            		\begin{enumerate}
                            			\def\labelenumi{\arabic{enumi}.}
                            			\tightlist
                            			\item El usuario accede a la sección de rúbricas.
                                        \item El usuario busca en la tabla la rúbrica que desee.
                            			\item El usuario pulsa el botón mostrar.
                                        \item El usuario visualiza el contenido de la rúbrica.
                                        \item El usuario pulsa el botón eliminar.
                                        \item El usuario confirma que desea eliminar.
                                    \end{enumerate}
                            		\end{enumerate} \\
		\textbf{Postcondición}        & \begin{itemize}
                                        \tightlist
		                                  \item La rúbrica es eliminada de la base de datos.
		                                  \item La aplicación redirige a la vista de rúbricas.
		                                  \item Mensaje: Rúbrica eliminada correctamente.
		                                 \end{itemize} \\
		\textbf{Excepciones}         & \begin{itemize}
                                        \tightlist
		                                  \item Error 404 - Rúbrica no encontrada.
		                                 \end{itemize} \\
		\textbf{Importancia}          & Alta \\
		\bottomrule
	\end{tabularx}
	\caption{CU-10 Eliminar rúbrica.}
	\label{tab:CU-10}
\end{table}

% Caso de Uso 11 -> Configurar correcciónes
\begin{table}[p]
	\centering
	\begin{tabularx}{\linewidth}{ p{0.21\columnwidth} p{0.71\columnwidth} }
		\toprule
		\textbf{CU-11}    & \textbf{Configurar corrección} \\
		\midrule
		\textbf{Versión}              & 1.0    \\
		\textbf{Autor}                & Pedro Antonio Abellaneda Canales \\
		\textbf{Requisitos asociados} & RF-5 \\
		\textbf{Descripción}          & Permite configurar una corrección a un usuario \\
		\textbf{Precondición}         & \begin{itemize}
                                        \tightlist
		                                  \item El usuario debe estar autenticado.
		                                  \item El usuario debe haber creado un \textit{prompt}.
		                                  \item El usuario debe haber importado una rúbrica.
		                                  \item Debe de existir un modelo cargado en local.
		                                 \end{itemize} \\
		\textbf{Acciones}             &
                            		\begin{enumerate}
                            			\def\labelenumi{\arabic{enumi}.}
                            			\tightlist
                            			\item El usuario accede a la sección de Correcciones.
                                        \item El usuario pulsa botón Nueva corrección.
                            			\item El usuario selecciona un \textit{prompt} de la lista.
                            			\item El usuario selecciona una rúbrica de la lista.
                            			\item El usuario carga un fichero comprimido con las tareas.
                                        \item El usuario introduce un nombre.
                                        \item El usuario selecciona un modelo de la lista.
                                        \item El usuario configura los parámetros del modelo.
                                        \item El usuario pulsa el botón Guardar configuración.
                            		\end{enumerate} \\
		\textbf{Postcondición}        & \begin{itemize}
                                        \tightlist
		                                  \item La aplicación redirige a la vista de correcciones.
		                                  \item Mensaje: Corrección creada correctamente.
		                                 \end{itemize} \\
		\textbf{Excepciones}         & \begin{itemize}
                                        \tightlist
		                                  \item La aplicación redirige a la vista de nueva corrección.
		                                  \item Mensaje: Error al crear la corrección.
		                                  \item Mensaje: Error en el campo Prompt: Este campo es requerido.
		                                  \item Mensaje: Error en el campo Rúbrica: Este campo es requerido.
		                                  \item Mensaje: Error en el campo Modelo: Este campo es requerido.
		                                 \end{itemize} \\
		\textbf{Importancia}          & Alta \\
		\bottomrule
	\end{tabularx}
	\caption{CU-11 Configurar corrección.}
	\label{tab:CU-11}
\end{table}

% Caso de Uso 12 -> Listar correcciónes
\begin{table}[p]
	\centering
	\begin{tabularx}{\linewidth}{ p{0.21\columnwidth} p{0.71\columnwidth} }
		\toprule
		\textbf{CU-12}    & \textbf{Listar correcciones} \\
		\midrule
		\textbf{Versión}              & 1.0    \\
		\textbf{Autor}                & Pedro Antonio Abellaneda Canales \\
		\textbf{Requisitos asociados} & RF-5.1 \\
		\textbf{Descripción}          & Permite visualizar las correcciones configuradas de un usuario \\
		\textbf{Precondición}         & \begin{itemize}
                                        \tightlist
		                                  \item El usuario debe estar autenticado.
		                                  \item El usuario debe de haber configurado una corrección.
		                                 \end{itemize} \\
		\textbf{Acciones}             &
                            		\begin{enumerate}
                            			\def\labelenumi{\arabic{enumi}.}
                            			\tightlist
                            			\item El usuario accede a la sección de Correcciones.
                                        \item El usuario pulsa botón Ver correcciones.
                            		\end{enumerate} \\
		\textbf{Postcondición}        & \begin{itemize}
                                        \tightlist
		                                  \item Se muestra la tabla con los campos y acciones disponibles para cada corrección.
		                                 \end{itemize} \\
		\textbf{Excepciones}         & \\
		\textbf{Importancia}          & Alta \\
		\bottomrule
	\end{tabularx}
	\caption{CU-12 Listar correcciones.}
	\label{tab:CU-12}
\end{table}

% Caso de Uso 13 -> Mostar configuración
\begin{table}[p]
	\centering
	\begin{tabularx}{\linewidth}{ p{0.21\columnwidth} p{0.71\columnwidth} }
		\toprule
		\textbf{CU-12}    & \textbf{Mostrar configuración} \\
		\midrule
		\textbf{Versión}              & 1.0    \\
		\textbf{Autor}                & Pedro Antonio Abellaneda Canales \\
		\textbf{Requisitos asociados} & RF-5.2 \\
		\textbf{Descripción}          & Permite visualizar los datos de configuración de una corrección \\
		\textbf{Precondición}         & \begin{itemize}
                                        \tightlist
		                                  \item El usuario debe estar autenticado.
		                                  \item El usuario debe de haber configurado una corrección.
		                                 \end{itemize} \\
		\textbf{Acciones}             &
                            		\begin{enumerate}
                            			\def\labelenumi{\arabic{enumi}.}
                            			\tightlist
                            			\item El usuario accede a la sección de Correcciones.
                                        \item El usuario pulsa botón Ver correcciones.
                                        \item El usuario busca la corrección que desee.
                                        \item El usuario pulsa botón Mostrar configuración, de la corrección que desee.
                            		\end{enumerate} \\
		\textbf{Postcondición}        & \begin{itemize}
                                        \tightlist
		                                  \item Se muestran los datos relacionados con la corrección seleccionada.
		                                 \end{itemize} \\
		\textbf{Excepciones}         & \\
		\textbf{Importancia}          & Alta \\
		\bottomrule
	\end{tabularx}
	\caption{CU-13 Mostrar configuración.}
	\label{tab:CU-13}
\end{table}


% Caso de Uso 14 -> Eliminar corrección
\begin{table}[p]
	\centering
	\begin{tabularx}{\linewidth}{ p{0.21\columnwidth} p{0.71\columnwidth} }
		\toprule
		\textbf{CU-14}    & \textbf{Eliminar corrección} \\
		\midrule
		\textbf{Versión}              & 1.0    \\
		\textbf{Autor}                & Pedro Antonio Abellaneda Canales \\
		\textbf{Requisitos asociados} & RF-5.3 \\
		\textbf{Descripción}          & Permite eliminar una corrección. \\
		\textbf{Precondición}         & \begin{itemize}
                                        \tightlist
		                                  \item El usuario debe estar autenticado.
		                                  \item El usuario debe de haber configurado una corrección.
		                                 \end{itemize} \\
		\textbf{Acciones}             &
                            		\begin{enumerate}
                            			\def\labelenumi{\arabic{enumi}.}
                            			\tightlist
                            			\item El usuario accede a la sección de Correcciones.
                                        \item El usuario pulsa botón Ver correcciones.
                                        \item El usuario busca la corrección que desee.
                                        \item El usuario pulsa botón Eliminar, de la corrección que desee.
                                        \item El usuario confirma que desea eliminar.
                            		\end{enumerate} \\
		\textbf{Postcondición}        & \begin{itemize}
                                        \tightlist
		                                  \item La aplicación redirige a la vista de correcciones.
		                                  \item Se eliminan los archivos relacionados con la corrección.
		                                  \item Mensaje: Corrección: <descripción>, eliminada correctamente.
		                                 \end{itemize} \\
		\textbf{Excepciones}         & \begin{itemize}
                                        \tightlist
		                                  \item Error 404 - Corrección no encontrada.
		                                 \end{itemize} \\
		\textbf{Importancia}          & Alta \\
		\bottomrule
	\end{tabularx}
	\caption{CU-14 Eliminar corrección.}
	\label{tab:CU-14}
\end{table}


% Caso de Uso 15 -> Ejecutar corrección
\begin{table}[p]
	\centering
	\begin{tabularx}{\linewidth}{ p{0.21\columnwidth} p{0.71\columnwidth} }
		\toprule
		\textbf{CU-15}    & \textbf{Ejecutar corrección} \\
		\midrule
		\textbf{Versión}              & 1.0    \\
		\textbf{Autor}                & Pedro Antonio Abellaneda Canales \\
		\textbf{Requisitos asociados} & RF-5.4 \\
		\textbf{Descripción}          & Permite ejecutar una corrección. \\
		\textbf{Precondición}         & \begin{itemize}
                                        \tightlist
		                                  \item El usuario debe estar autenticado.
		                                  \item El usuario debe de haber configurado una corrección.
		                                  \item El modelo seleccionado para la corrección debe estar cargado en local.
		                                 \end{itemize} \\
		\textbf{Acciones}             &
                            		\begin{enumerate}
                            			\def\labelenumi{\arabic{enumi}.}
                            			\tightlist
                            			\item El usuario accede a la sección de Correcciones.
                                        \item El usuario pulsa botón Ver correcciones.
                                        \item El usuario busca la corrección que desee.
                                        \item El usuario pulsa botón Ejecutar, de la corrección que desee.
                            		\end{enumerate} \\
		\textbf{Postcondición}        & \begin{itemize}
                                        \tightlist
		                                  \item Se envían los datos y las tareas de la corrección ejecutada al modelo
		                                  \item Mensaje: Corrección en proceso. Revisa el resultado más tarde.
		                                 \end{itemize} \\
		\textbf{Excepciones}         & \begin{itemize}
                                        \tightlist
		                                  \item Mensaje: El modelo seleccionado para esta corrección no está cargado en Ollama.
		                                 \end{itemize} \\
		\textbf{Importancia}          & Alta \\
		\bottomrule
	\end{tabularx}
	\caption{CU-15 Ejecutar corrección.}
	\label{tab:CU-15}
\end{table}

% Caso de Uso 16 -> Descargar resultado
\begin{table}[p]
	\centering
	\begin{tabularx}{\linewidth}{ p{0.21\columnwidth} p{0.71\columnwidth} }
		\toprule
		\textbf{CU-16}    & \textbf{Descargar resultado} \\
		\midrule
		\textbf{Versión}              & 1.0    \\
		\textbf{Autor}                & Pedro Antonio Abellaneda Canales \\
		\textbf{Requisitos asociados} & RF-5.5 \\
		\textbf{Descripción}          & Permite ejecutar una corrección. \\
		\textbf{Precondición}         & \begin{itemize}
                                        \tightlist
		                                  \item El usuario debe estar autenticado.
		                                  \item El usuario debe de haber configurado una corrección.
		                                  \item El usuario debe de haber ejecutado una corrección.
		                                 \end{itemize} \\
		\textbf{Acciones}             &
                            		\begin{enumerate}
                            			\def\labelenumi{\arabic{enumi}.}
                            			\tightlist
                            			\item El usuario accede a la sección de Correcciones.
                                        \item El usuario pulsa botón Ver correcciones.
                                        \item El usuario busca la corrección que desee.
                                        \item El usuario pulsa botón Descargar último resultado, de la corrección que desee.
                            		\end{enumerate} \\
		\textbf{Postcondición}        & \begin{itemize}
                                        \tightlist
		                                  \item Se descarga el resultado de la última corrección ejecutada.
		                                 \end{itemize} \\
		\textbf{Excepciones}         &  \\
		\textbf{Importancia}          & Alta \\
		\bottomrule
	\end{tabularx}
	\caption{CU-16 Descargar corrección.}
	\label{tab:CU-16}
\end{table}
