\apendice{Especificación de Requisitos}

\section{Introducción}

Una muestra de cómo podría ser una tabla de casos de uso:

\section{Objetivos generales}

Este trabajo se ha realizado persiguiendo el cumplimiento de los siguientes \textbf{objetivos:}

\begin{itemize}
    \item Desarrollar un \textit{sistema software} que permita a los usuarios la corrección de ejercicios de programación.
    \item El usuario debe estar \textit{registrado} para acceder al sistema iniciando sesión.
    \item Los usuarios \textit{importarán} sus propias rúbricas en formato Markdown
    \item Los usuarios crearán sus propios prompts.
    \item La evaluación de los ejercicios se realizará mediante \textit{modelos de lenguaje LLM}.
    \item Los usuarios podrán configurar aspectos determinados del LLM usado.
    
\end{itemize}

\section{Catálogo de requisitos}
Requisitos funcionales y no funcionales para cumplir con los objetivos generales

\subsection{Requisitos funcionales}
\begin{itemize}
    \item \textbf{RF-1 - Registro de usuarios.} Los usuarios podrán registrarse mediante nombre, correo electrónico y contraseña.
    \item \textbf{RF-2 - Login de usuarios.} Los usuarios deberán iniciar sesión para usar la aplicación.
    \item \textbf{RF-3 - Importar rúbricas.} Los usuarios podrán importar sus propias rúbricas en formato Markdown.
    \begin{itemize}
        \item \textbf{RF-3.1 - Listar rúbricas.} Los usuarios podrán consultar sus rúbricas importadas.
        \item \textbf{RF-3.2 - Consultar rubricas.} Los usuarios podrán visualizar los detalles de una rúbrica concreta.
        \item \textbf{RF-3.3 - Eliminar rúbricas.} Los usuarios podrán eliminar una rúbrica importada.
    \end{itemize}
    \item \textbf{RF-4 Crear prompts.} Los usuarios crearán sus propios prompts.
    \begin{itemize}
        \item \textbf{RF-4.1 - Listar prompts.} Los usuarios podrán consultar sus prompts creados.
        \item \textbf{RF-4.2 - Consultar prompts.} Los usuarios podrán visualizar un prompt concreto.
        \item \textbf{RF-4.3 - Eliminar prompts.} Los usuarios podrán eliminar un prompt.
    \end{itemize}
    \item \textbf{RF-5 - Configurar LLM.} Los usuarios podrán configurar aspectos determinados del LLM seleccionado.
    \begin{itemize}
        \item \textbf{RF-5.1 - Seleccionar LLM.} Los usuarios seleccionarán el LLM a usar.
        \item \textbf{RF-5.2 - Parametrizar LLM.} Los usuarios podrán modificar determinados parámetros del LLM seleccionado.
    \end{itemize}
\end{itemize}

\section{Especificación de requisitos}

Diagramas de casos de uso y detalles sobre los mismos.

\imagen{caso_de_uso_general}{Diagrama de caso de uso general}

% Caso de Uso 1 -> Registrar usuario.
\begin{table}[p]
	\centering
	\begin{tabularx}{\linewidth}{ p{0.21\columnwidth} p{0.71\columnwidth} }
		\toprule
		\textbf{CU-1}    & \textbf{Registrar usuario}\\
		\toprule
		\textbf{Versión}              & 1.0    \\
		\textbf{Autor}                & Alumno \\
		\textbf{Requisitos asociados} & RF-xx, RF-xx \\
		\textbf{Descripción}          & La descripción del CU \\
		\textbf{Precondición}         & Precondiciones (podría haber más de una) \\
		\textbf{Acciones}             &
		\begin{enumerate}
			\def\labelenumi{\arabic{enumi}.}
			\tightlist
			\item Pasos del CU
			\item Pasos del CU (añadir tantos como sean necesarios)
		\end{enumerate}\\
		\textbf{Postcondición}        & Postcondiciones (podría haber más de una) \\
		\textbf{Excepciones}          & Excepciones \\
		\textbf{Importancia}          & Alta o Media o Baja... \\
		\bottomrule
	\end{tabularx}
	\caption{CU-1 Nombre del caso de uso.}
\end{table}



