\apendice{Especificación de diseño}

\section{Introducción}

En esta sección se describe el proceso de diseño de la aplicación \textit{IAGScore}, orientado a dar cumplimiento a los objetivos y requisitos definidos con anterioridad.  
Se especifican los datos que utilizará la aplicación, así como su estructura procedimental y la arquitectura del sistema.

\section{Diseño de datos}

En este proyecto se han desarrollado seis aplicaciones, organizadas con el objetivo de separar de forma coherente la lógica y las funcionalidades del sistema, de acuerdo con los resultados esperados.  
Existe una aplicación principal denominada \textit{iagscore}, encargada de la gestión y configuración general del proyecto, y otras cinco aplicaciones denominadas \textit{accounts}, \textit{core}, \textit{prompts}, \textit{rubrics} y \textit{corrections}.

El diagrama entidad-relación que se presenta a continuación refleja las entidades y relaciones que forman parte del modelo de datos persistente de la aplicación. En este sentido, se han incluido únicamente  los modelos propios definidos en las aplicaciones que generan tablas en la base de datos: \textit{accounts}, \textit{prompts}, \textit{rubrics} y \textit{corrections}. 

Las aplicaciones \textit{iagscore} y \textit{core}, aunque esenciales para la configuración, organización y funcionamiento general del sistema, no aportan entidades directamente al esquema relacional, ya que no definen modelos. Por este motivo, no se representan en el diagrama, aunque su funcionalidad es fundamental para la arquitectura global del proyecto.

\subsection{Diagrama E/R}

\imagen{diagrama_ER}{Diagrama E/R}
\newpage
\subsection{Diagrama relacional}
\imagen{diagrama_relacional}{Diagrama Relacional}

\section{Diseño arquitectónico}

\textit{Django} permite acelerar el desarrollo de aplicaciones web seguras y escalables gracias a su enfoque \textit{batteries-included}~\cite{web:django_contrib}, que incluye componentes como un \textit{ORM}, sistema de autenticación, administración automática y soporte nativo para pruebas.

El sistema está construido siguiendo el patrón arquitectónico Modelo-Vista-\textit{Template} (\textit{MVT}), propio de \textit{Django}, que permite una separación clara de responsabilidades y facilita el mantenimiento y escalabilidad de la aplicación.

En este patrón:

\begin{itemize}
  \item \textbf{Modelo (\textit{Model})}: define la estructura de datos y la lógica para acceder y manipular la base de datos.
  \item \textbf{Vista (\textit{View})}: gestiona la lógica de negocio, procesa las peticiones del usuario y devuelve respuestas, normalmente utilizando plantillas para generar la presentación.
  \item \textbf{Plantilla (\textit{Template})}: define cómo se presenta la información, generando el contenido \textit{HTML} que la vista devuelve al usuario.
\end{itemize}

\imagen{mvt_django}{Diagrama \textit{MVT} de \textit{Django}}

Sobre esta base, el sistema se despliega en una arquitectura de microservicios contenerizados usando \textit{Docker}. Esta estructura permite el despliegue y escalado sencillo de los distintos componentes que conforman la aplicación, así como su separación lógica y técnica.

A continuación, se describen los principales servicios definidos en el archivo \texttt{docker-compose.yaml}:

\begin{itemize}
  \item \textbf{\textit{web}}: contenedor principal que ejecuta el servidor de \textit{Django}. Es el punto de entrada del sistema y gestiona tanto las peticiones de los usuarios como la lógica del negocio.

  \item \textbf{\emph{postgres\_db}}: servicio de base de datos relacional utilizando \textit{PostgreSQL}. Gestiona el almacenamiento persistente de la información del sistema, incluyendo usuarios, correcciones, rúbricas, \textit{prompts}, etc.

  \item \textbf{\textit{redis}}: sistema utilizado como \textit{backend} para la gestión de la cola de tareas asíncronas con \textit{Celery}.

  \item \textbf{\textit{celery}}: servicio de ejecución en segundo plano para tareas intensivas o que requieren tiempo de procesamiento prolongado. Utiliza \textit{Redis} como intermediario y permite delegar tareas como la corrección con \textit{LLM}.

  \item \textbf{\textit{ollama}}: servicio de modelos de lenguaje basado en la imagen oficial de \textit{Ollama}. Permite interactuar con modelos \textit{LLM} (\textit{Large Language Models}) directamente dentro de la infraestructura del sistema, sin necesidad de servicios externos.

\end{itemize}

Todos los servicios están conectados mediante una red interna personalizada (\texttt{app-network}), que garantiza la comunicación segura y eficiente entre ellos. Se utilizan volúmenes para asegurar la persistencia de datos relevantes tareas cargadas por los usuarios o las respuestas del modelo \textit{LLM}.

% diagrama de despliegue con los servicios Docker
\imagen{dg_despliegue}{Diagrama de despliegue de la arquitectura Docker}

\subsection{Estructura de Paquetes del Sistema}

El sistema ha sido desarrollado como una única aplicación web utilizando el \textit{framework} \textbf{\emph{Django}}.

Se ha seguido un \textbf{enfoque basado en características funcionales}, organizando el sistema en paquetes que agrupan todos los elementos relacionados con una misma funcionalidad (como autenticación, correcciones, rúbricas, etc.).

Esta elección arquitectónica promueve una \textbf{alta cohesión} dentro de cada paquete y un \textbf{bajo acoplamiento} entre ellos, en línea con los principios del diseño modular.

Los distintos diagramas de paquetes incluidos a continuación reflejan esta organización, permitiendo visualizar la estructura jerárquica interna y las dependencias funcionales de cada una de las aplicaciones que componen el sistema.

\subsubsection{Aplicación \textit{accounts}}

La aplicación \texttt{\textit{accounts}} gestiona el modelo de usuario personalizado. Su diagrama de paquetes muestra la estructura interna de los módulos relacionados con la gestión de usuarios.

\imagen{pq_accounts}{Diagrama de paquetes \textit{accounts}}

\subsubsection{Aplicación \textit{core}}

La aplicación \texttt{\textit{core}} contiene componentes compartidos por el resto del sistema y la configuración del sistema de autenticación.

\imagen{pk_core}{Diagrama de paquetes \textit{core}}

\subsubsection{Aplicación \textit{prompts}}

La aplicación \texttt{\textit{prompts}} se encarga de la creación, edición y visualización de los \textit{prompts}. El siguiente diagrama muestra su estructura modular y los componentes principales implicados.

\imagen{pk_prompts}{Diagrama de paquetes \textit{prompts}}

\subsubsection{Aplicación \textit{rubrics}}

La aplicación \texttt{\textit{rubrics}} gestiona las rúbricas utilizadas en los procesos de corrección. Su organización interna se detalla en el siguiente diagrama de paquetes.

\imagen{pk_rubrics}{Diagrama de paquetes \textit{rubrics}}

\subsubsection{Aplicación \textit{corrections}}

La aplicación \texttt{\textit{corrections}} es responsable de la configuración y ejecución de las correcciones automáticas, integrando \textit{prompts}, rúbricas, señales y tareas asíncronas. El siguiente diagrama muestra la disposición de sus módulos funcionales.

\imagen{pq_corrections}{Diagrama de paquetes \textit{corrections}}

El siguiente diagrama proporciona una visión general de la estructura de paquetes del sistema, mostrando las aplicaciones principales y sus interacciones.

\imagen{paquetes_completo}{Diagrama de paquetes completo}

\subsection{Gestión de la Persistencia con el ORM de Django}

Una de las piezas clave del sistema es el \textit{Object-Relational Mapper} (ORM) que proporciona \textit{Django}, utilizado como capa de abstracción entre los modelos del dominio y la base de datos relacional (\textit{PostgreSQL}). 

Cada modelo definido en el sistema representa una tabla, y las relaciones entre modelos se traducen directamente en claves foráneas. Esta representación se define de forma estructurada mediante clases en \textit{Python}, y permite una gestión completamente tipada, validada y controlada del esquema relacional.

El \textit{ORM} permite:

\begin{itemize}
  \item Crear y modificar el esquema de base de datos utilizando migraciones (\texttt{makemigrations} y \texttt{migrate}).
  \item Definir restricciones, relaciones y validaciones desde el propio modelo.
  \item Interactuar con los datos mediante una API de alto nivel sin necesidad de escribir SQL.
  \item Integrar de forma directa el modelo de datos con componentes del sistema como vistas, serializadores o la administración web.
\end{itemize}

Gracias a este enfoque, los modelos del sistema no sólo representan la estructura de la base de datos, sino que encapsulan también la lógica de negocio básica asociada a cada entidad (como métodos personalizados o validaciones), fomentando un diseño centrado en el dominio~\cite{web:django_queries}.

A continuación, se muestra un ejemplo representativo del modelo que se implementa en la aplicación \texttt{corrections} (\texttt{Correction}):

\begin{verbatim}
class Correction(models.Model):
    prompt = models.ForeignKey(Prompt, on_delete=models.CASCADE)
    rubric = models.ForeignKey(Rubric, on_delete=models.CASCADE)
    user = models.ForeignKey(User, on_delete=models.CASCADE)
    date = models.DateTimeField(auto_now_add=True)
    description = models.TextField()
\end{verbatim}

Esta arquitectura favorece una clara separación de responsabilidades entre la lógica de negocio, las operaciones de persistencia y la presentación, en línea con el patrón MVT.

\section{Diseño procedimental}

En esta sección se presentan los diagramas de secuencia que describen el flujo de interacción entre los distintos componentes del sistema durante la ejecución de operaciones clave. Estos diagramas ayudan a visualizar el comportamiento dinámico de la aplicación, mostrando el orden de los mensajes intercambiados entre actores, vistas, modelos, tareas en segundo plano y bases de datos.

\subsection{Diagrama de secuencias: crear nuevo \textit{prompt}}

Este diagrama representa el proceso seguido por el sistema cuando un usuario crea un nuevo \textit{prompt}. Incluye la validación del formulario, la persistencia en base de datos y la respuesta adecuada al cliente.

\imagen{dg_prompt}{Diagrama de secuencias: crear nuevo \textit{prompt}}

\subsection{Diagrama de secuencias: crear nueva rúbrica}

Este diagrama detalla los pasos realizados al importar una nueva rúbrica desde un archivo \texttt{.md}. Se ilustra cómo se valida el archivo, se comprueba su unicidad, y se guarda en la base de datos si es válido.

\imagen{ds_rubric}{Diagrama de secuencia: crear nueva rúbrica}

\subsection{Diagrama de secuencias: crear nueva corrección}

Este diagrama muestra la creación de una nueva corrección, asociando los elementos necesarios como rúbrica, \textit{prompt} y tareas. También se realiza la validación de los datos antes de su almacenamiento.

\imagen{ds_new_correction}{Diagrama de secuencias: crear nueva corrección}

\subsection{Diagrama de secuencias: Ejecutar corrección}

Este diagrama describe el proceso de ejecución de una corrección utilizando un modelo LLM. Incluye la preparación de datos, el uso de una tarea en segundo plano mediante Celery, y la interacción con el sistema de archivos y el modelo \textit{Ollama}.

\imagen{ds_run_correction}{Diagrama de secuencias: ejecutar corrección}
