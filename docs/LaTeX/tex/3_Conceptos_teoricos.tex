\capitulo{3}{Conceptos teóricos}

En este apartado se describen los componentes principales que permiten llevar 
a cabo la evaluación automática de tareas mediante modelos de lenguaje. 
Se comienza explicando qué son los modelos de lenguaje de gran tamaño (LLM) 
y cómo funcionan. A continuación, se introduce LangChain como una herramienta 
útil para integrar estos modelos en flujos de trabajo personalizados. 
Seguidamente, se presenta Ollama como solución para ejecutar modelos de forma 
local, priorizando la privacidad y el control. Finalmente, se hace una mención 
al prompt engineering, un enfoque que permite mejorar o personalizar las 
respuestas de los modelos en función de cómo se les formula la entrada, y 
que puede tener relevancia en el desarrollo de este tipo de sistemas.

\section{Modelos de lenguaje grandes (LLM)}

Los LLM son sistemas de IA utilizados para modelar y procesar el lenguaje humano.
Se llaman LLM porque son modelos estadísticos de lenguaje entrenados con volúmenes 
masivos de texto, y tienen una arquitectura muy grande capaz de capturar matices 
lingüísticos complejos que los modelos más pequeños no pueden.

La tecnología subyacente de los LLM se basa en una arquitectura de red neuronal llamada 
Transformer, propuesta por primera vez en el artículo ``Attention Is All You Need''~\cite{vaswani2017}. 
contrutendo una arquitectura basada en mecanismos de atención, sin recurrir a 
redes recurrentes o convolucionales.

\section{Langchain }

Las secciones se incluyen con el comando section.

\section{Ollama}

Las secciones se incluyen con el comando section.

\section{Prompt engineering}

Las secciones se incluyen con el comando section.

\subsubsection{Subsubsecciones}

Y subsecciones. 


\section{Referencias}

Las referencias se incluyen en el texto usando cite~\cite{wiki:latex}. Para citar webs, artículos o libros~\cite{koza92}, si se desean citar más de uno en el mismo lugar~\cite{bortolot2005, koza92}.


\section{Imágenes}

Se pueden incluir imágenes con los comandos standard de \LaTeX, pero esta plantilla dispone de comandos propios como por ejemplo el siguiente:

\imagen{escudoInfor}{Autómata para una expresión vacía}{.5}



\section{Listas de items}

Existen tres posibilidades:

\begin{itemize}
	\item primer item.
	\item segundo item.
\end{itemize}

\begin{enumerate}
	\item primer item.
	\item segundo item.
\end{enumerate}

\begin{description}
	\item[Primer item] más información sobre el primer item.
	\item[Segundo item] más información sobre el segundo item.
\end{description}
	
\begin{itemize}
\item 
\end{itemize}

\section{Tablas}

Igualmente se pueden usar los comandos específicos de \LaTeX o bien usar alguno de los comandos de la plantilla.

\tablaSmall{Herramientas y tecnologías utilizadas en cada parte del proyecto}{l c c c c}{herramientasportipodeuso}
{ \multicolumn{1}{l}{Herramientas} & App AngularJS & API REST & BD & Memoria \\}{ 
HTML5 & X & & &\\
CSS3 & X & & &\\
BOOTSTRAP & X & & &\\
JavaScript & X & & &\\
AngularJS & X & & &\\
Bower & X & & &\\
PHP & & X & &\\
Karma + Jasmine & X & & &\\
Slim framework & & X & &\\
Idiorm & & X & &\\
Composer & & X & &\\
JSON & X & X & &\\
PhpStorm & X & X & &\\
MySQL & & & X &\\
PhpMyAdmin & & & X &\\
Git + BitBucket & X & X & X & X\\
Mik\TeX{} & & & & X\\
\TeX{}Maker & & & & X\\
Astah & & & & X\\
Balsamiq Mockups & X & & &\\
VersionOne & X & X & X & X\\
} 
