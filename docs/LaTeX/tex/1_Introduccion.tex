\capitulo{1}{Introducción}

Evaluar las tareas de programación que entregan los estudiantes no solo implica verificar que el código funcione correctamente, sino también valorar aspectos como la calidad del diseño, la claridad, la eficiencia y el cumplimiento de criterios específicos establecidos por el docente.

Tradicionalmente, la corrección manual de estas tareas es una labor que demanda una gran inversión de tiempo y esfuerzo por parte del profesorado, especialmente en cursos con gran número de estudiantes. Esta situación limita la capacidad de proporcionar una retroalimentación rápida, detallada y personalizada, aspectos fundamentales para el aprendizaje eficaz.

En los últimos años, el avance en modelos de lenguaje a gran escala (\textit{LLM}) ha abierto nuevas posibilidades en el campo de la docencia. Estos modelos tienen la capacidad de procesar y comprender tanto texto como código, lo que los convierte en herramientas prometedoras para asistir en la corrección de tareas, ofreciendo comentarios elaborados que van más allá de una simple validación funcional.

Este proyecto se centra en el desarrollo de una aplicación web que integra el potencial de los \textit{LLM} para realizar evaluaciones automatizadas basadas en criterios definidos por el usuario, mediante la incorporación de rúbricas personalizadas y \textit{prompts} específicos diseñados por los usuarios. La aplicación busca facilitar la labor docente, permitiendo una gestión eficiente de las correcciones y mejorando la calidad de la retroalimentación brindada a los estudiantes.

De esta forma, la herramienta propuesta contribuye a modernizar la enseñanza de la programación, promoviendo una evaluación más ágil, objetiva y enriquecedora, que favorezca tanto al profesorado como al alumnado, sin sustituir la necesaria supervisión humana en este proceso.

\section{Estructura de la memoria}

La memoria sigue la siguiente estructura:

\begin{itemize}
    \item \textbf{Introducción:} breve descripción del problema a resolver y la solución propuesta.
    \item \textbf{Objetivos del proyecto:} exposición de los objetivos que persigue el proyecto.
    \item \textbf{Conceptos teóricos:} breve explicación de los conceptos teóricos.
    \item \textbf{Técnicas y herramientas:} listado de técnicas metodológicas y herramientas utilizadas.
    \item \textbf{Aspectos relevantes del desarrollo:} exposición de aspectos destacables que tuvieron lugar durante la realización del proyecto.
    \item \textbf{Trabajos relacionados:} pequeño resumen de los trabajos y proyectos ya realizados en el campo del proyecto en curso.
    \item \textbf{Conclusiones y líneas de trabajo futuras:} conclusiones obtenidas tras la realización del proyecto.
\end{itemize}

Junto a la memoria se proporcionan los siguientes anexos:

\begin{itemize}
    \item \textbf{Plan del proyecto software:} planificación temporal y estudio de viabilidad del proyecto.
    \item \textbf{Especificación de requisitos del software:} se describe la fase de análisis; los objetivos generales, el catálogo de requisitos del sistema y la especificación de requisitos funcionales y no funcionales.
    \item \textbf{Especificación de diseño:} se describe la fase de diseño; el ámbito del software, el diseño de datos, el diseño procedimental y el diseño arquitectónico.
    \item \textbf{Manual del programador:} recoge los aspectos más relevantes relacionados con el código fuente.
    \item \textbf{Manual de usuario:} guía para el correcto manejo de la aplicación IAGScore.
\end{itemize}

Además, se incluyen los siguientes materiales complementarios accesibles a través de Internet:

\begin{itemize}
    \item \textbf{Repositorio:} el código fuente y documentación están disponibles en el repositorio oficial de \href{https://github.com/pac1006/IAGScore}{IAGScore en GitHub}.
    \item \textbf{Gestión del proyecto:} realizada con \href{https://zube.io/miorganizacion/iagscore/w/workspace-1/kanban}{Zube}.
    \item \textbf{Vídeo de presentación:} \href{https://universidaddeburgos-my.sharepoint.com/:v:/g/personal/pac1006_alu_ubu_es/EfB4_46cvc1Ltzs4VFGbCFkBWj3Dm_AiB9mt6hpTRJkKeQ?nav=eyJyZWZlcnJhbEluZm8iOnsicmVmZXJyYWxBcHAiOiJPbmVEcml2ZUZvckJ1c2luZXNzIiwicmVmZXJyYWxBcHBQbGF0Zm9ybSI6IldlYiIsInJlZmVycmFsTW9kZSI6InZpZXciLCJyZWZlcnJhbFZpZXciOiJNeUZpbGVzTGlua0NvcHkifX0&e=3IHW2I}{presentación} general del proyecto y sus objetivos.
    \item \textbf{Vídeo de demostración:} \href{https://universidaddeburgos-my.sharepoint.com/:v:/g/personal/pac1006_alu_ubu_es/ETGJTbAviu5NrIwAoXvITHYBXlwclYbDOcRTdR0pd7QDdw?nav=eyJyZWZlcnJhbEluZm8iOnsicmVmZXJyYWxBcHAiOiJPbmVEcml2ZUZvckJ1c2luZXNzIiwicmVmZXJyYWxBcHBQbGF0Zm9ybSI6IldlYiIsInJlZmVycmFsTW9kZSI6InZpZXciLCJyZWZlcnJhbFZpZXciOiJNeUZpbGVzTGlua0NvcHkifX0&e=vbhU2M}{demostración} funcional de la aplicación en uso.
\end{itemize}
