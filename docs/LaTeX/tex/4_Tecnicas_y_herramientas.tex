\capitulo{4}{Técnicas y herramientas}
Esta parte de la memoria tiene como objetivo presentar las técnicas metodológicas y las herramientas de desarrollo que se han utilizado para llevar a cabo el proyecto. Si se han estudiado diferentes alternativas de metodologías, herramientas, bibliotecas se puede hacer un resumen de los aspectos más destacados de cada alternativa, incluyendo comparativas entre las distintas opciones y una justificación de las elecciones realizadas. 
No se pretende que este apartado se convierta en un capítulo de un libro dedicado a cada una de las alternativas, sino comentar los aspectos más destacados de cada opción, con un repaso somero a los fundamentos esenciales y referencias bibliográficas para que el lector pueda ampliar su conocimiento sobre el tema.

\section{Metodologías}

\subsection{Scrum}

Scrum es un marco de trabajo ágil, líder en el desarrollo de software y la gestión de proyectos, 
que se distingue por su flexibilidad. Organiza el trabajo en sprints, iteraciones cortas de una a cuatro 
semanas, cada una entregando un incremento funcional del producto. 
Al final de cada sprint, se evalúa el resultado para detectar mejoras 
y ajustar prioridades según las necesidades del proyecto. Los requisitos, que pueden cambiar durante el 
desarrollo, se recogen en la pila del producto, mientras que las tareas específicas de cada iteración se 
detallan en la pila del sprint. Este enfoque incremental permite solapar fases del proyecto, optimizando 
tiempos y facilitando la adaptación a nuevos requerimientos. Scrum promueve la calidad a través del
aprendizaje continuo del equipo, generando entregas tempranas de productos utilizables y asegurando una 
mejora constante alineada con las necesidades del usuario~\cite{wiki:Scrum}.

\subsection{Kanban}
Kanban es un método ágil que se centra en la visualización del flujo de trabajo y la gestión del
trabajo en curso. Utiliza un tablero representando las tareas en una serie de tarjetas, permitiendo a los
miembros del equipo ver el progreso y las prioridades del trabajo.
Kanban Se basa en principios como la limitación del trabajo en curso, la mejora continua y la
adaptación a los cambios, es flexible y se puede aplicar a diferentes tipos de proyectos y equipos,
lo que lo convierte en una herramienta valiosa para la gestión de proyectos en entornos ágiles~\cite{wiki:Kanban}.

\subsection{Scrumban}
Scrumban es una metodología híbrida que combina elementos de Scrum y Kanban.
Se basa en la estructura de Scrum, con sprints y roles definidos, pero incorpora la flexibilidad de Kanban
en la gestión del flujo de trabajo.
Esta metodología es especialmente útil para equipos que buscan una transición suave entre Scrum y Kanban,
o para aquellos que desean aprovechar lo mejor de ambos enfoques.
Scrumban se centra en la mejora continua, la visualización del trabajo y la colaboración del equipo,
lo que lo convierte en una opción popular para la gestión de proyectos ágiles~\cite{web:scrumban}.

\subsubsection{Gitflow}

Gitflow~\cite{web:Gitflow} es un modelo de ramificación para el uso de Git que define una 
estructura clara para gestionar el desarrollo de software en proyectos colaborativos. 
Organiza el trabajo en ramas específicas con roles definidos, 
facilitando la escalabilidad y la integración continua.

Las ramas principales en Gitflow son:
\begin{itemize}
    \item \textbf{\texttt{main}}: Contiene la versión estable del proyecto, lista para producción.
    \item \textbf{\texttt{develop}}: Refleja la versión en desarrollo, integrando nuevas funcionalidades.
    \item \textbf{\texttt{feature/*}}: Ramas temporales para desarrollar nuevas características.
    \item \textbf{\texttt{release/*}}: Prepara una versión para producción, permitiendo ajustes menores.
    \item \textbf{\texttt{hotfix/*}}: Corrige errores críticos en la rama \texttt{main}.
\end{itemize}

Gitflow asegura un flujo ordenado, minimizando conflictos y garantizando estabilidad en la rama \texttt{main}.

\section{Patrones de diseño}
\subsection{Model-View-Template}

El patrón de diseño MVT (\textit{Model-View-Template}) es la arquitectura que sigue el framework Django para 
desarrollar aplicaciones web directamente relacionada con el patrón MVC.

Los componentes detallados del patrón MVT son:

\begin{itemize}
    \item \textbf{Model (Modelo)}: Representa la capa de acceso a datos. Define la estructura de las tablas de la 
    base de datos y permite interactuar con esos datos. En Django, los modelos se definen como clases que heredan 
    de \texttt{django.db.models.Model}, y permiten realizar operaciones de consulta, inserción, actualización o 
    borrado sin necesidad de escribir directamente SQL.
    
    \item \textbf{View (Vista)}: Esta capa contiene la lógica de negocio de la aplicación. Las vistas reciben 
    peticiones HTTP, procesan los datos necesarios y devuelven una respuesta adecuada.
    
    \item \textbf{Template (Plantilla)}: Define la presentación de los datos, lo que ve el usuario. 
    Django utiliza su propio motor de plantillas para combinar datos dinámicos con HTML. 
    
\end{itemize}

Una diferencia clave entre el patrón MVT de Django y el clásico MVC es que el propio framework gestiona 
internamente la lógica de enrutamiento. Al recibir peticiones del usuario, se decide a través del sistema 
de URLs definido qué vista debe ejecutarse, lo que elimina la necesidad de un controlador explícito. 
En este sentido, las vistas en Django cumplen una función similar al controlador 
en otros frameworks, ya que contienen la lógica que determina qué datos recuperar y cómo deben procesarse 
antes de enviarse a la plantilla correspondiente~\cite{web:djangoMVT}.

\section{Control de versiones}

En este proyecto, se ha optado por Git~\cite{web:Git} debido a la experiencia previa adquirida en asignaturas del grado, 
en contraste con otras alternativas como Apache Subversion~\cite{web:Subversion}.

\subsubsection{Git}

Git es un sistema de control de versiones distribuido y una de las herramientas 
más extendidas en el mercado para este propósito. Entre sus características más destacadas se 
encuentran su eficiencia para manejar proyectos de gran tamaño, la facilidad para crear y fusionar 
ramas (\textit{branches}), y su compatibilidad con diversas plataformas y servicios como GitHub, GitLab o Bitbucket.


\section{Alojamiento del repositorio}

Existen diversas posibilidades relacionadas con el alojamiento de repositorios:

\begin{itemize}
    \item GitLab~\cite{web:GitLab}
    \item BitBucket~\cite{web:Bitbucket}
    \item GitHub~\cite{web:GitHub}
\end{itemize}

\subsubsection{GitHub}

GitHub es una plataforma en línea para alojar repositorios Git, que 
facilita la colaboración, el control de versiones y la gestión de proyectos de software. 
Entre sus características destacan la gestión de ramas y \textit{pull requests}, el seguimiento 
de incidencias (\textit{issues}), y la integración con herramientas de integración continua como 
GitHub Actions.

Se ha seleccionado GitHub al estar familiarizado con la plataforma debido al uso de 
esta herramienta en múltiples asignaturas.

\section{Gestión del proyecto}

\subsubsection{Zube}
