\capitulo{4}{Técnicas y herramientas}

\section{Metodologías}

\subsection{Scrum}

Scrum es un marco de trabajo ágil, líder en el desarrollo de software y la gestión de proyectos, 
que se distingue por su flexibilidad. Organiza el trabajo en sprints, iteraciones cortas de una a cuatro 
semanas, cada una entregando un incremento funcional del producto. 
Al final de cada sprint, se evalúa el resultado para detectar mejoras 
y ajustar prioridades según las necesidades del proyecto. Los requisitos, que pueden cambiar durante el 
desarrollo, se recogen en la pila del producto, mientras que las tareas específicas de cada iteración se 
detallan en la pila del sprint. Este enfoque incremental permite solapar fases del proyecto, optimizando 
tiempos y facilitando la adaptación a nuevos requerimientos. Scrum promueve la calidad a través del
aprendizaje continuo del equipo, generando entregas tempranas de productos utilizables y asegurando una 
mejora constante alineada con las necesidades del usuario~\cite{wiki:Scrum}.

\subsection{Kanban}
Kanban es un método ágil que se centra en la visualización del flujo de trabajo y la gestión del
trabajo en curso. Utiliza un tablero representando las tareas en una serie de tarjetas, permitiendo a los
miembros del equipo ver el progreso y las prioridades del trabajo.
Kanban Se basa en principios como la limitación del trabajo en curso, la mejora continua y la
adaptación a los cambios, es flexible y se puede aplicar a diferentes tipos de proyectos y equipos,
lo que lo convierte en una herramienta valiosa para la gestión de proyectos en entornos ágiles~\cite{wiki:Kanban}.

\subsection{Scrumban}
Scrumban es una metodología híbrida que combina elementos de Scrum y Kanban.
Se basa en la estructura de Scrum, con sprints y roles definidos, pero incorpora la flexibilidad de Kanban
en la gestión del flujo de trabajo.
Esta metodología es especialmente útil para equipos que buscan una transición suave entre Scrum y Kanban,
o para aquellos que desean aprovechar lo mejor de ambos enfoques.
Scrumban se centra en la mejora continua, la visualización del trabajo y la colaboración del equipo,
lo que lo convierte en una opción popular para la gestión de proyectos ágiles~\cite{web:scrumban}.
